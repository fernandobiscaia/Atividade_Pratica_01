\documentclass{article}
\usepackage[utf8]{inputenc}

\title{Atividade Prática 01}
\author{Fernando Biscaia }
\date{November 2021}

\begin{document}

\maketitle

\section{Quais as diferenças na estrutura da rede IPÊ de 2016 (slide 8) para 2020/2021?}

Existe mais conectividade com ligação direta entre mais cidades, além de maior velocidade de rede.

\section{Qual a diferença entre Web e Internet?}

A internet é uma rede que conecta milhões de computadores pelo mundo, enquanto a web é uma das várias ferramentas de acesso a essa rede. É a internet que provê serviços como: e-mail, FTP e troca de mensagens instantâneas. A web usa o protocolo HTTP para promover essa transferência de informações e depende de browsers (navegadores como Internet Explorer e Chrome) para apresentar tudo isso ao internauta, permitindo que ele clique em links que levam a arquivos hospedados em outros computadores.

\section{Quais orgãos administram o ponto br (.br) para além do slide 17?}

CEPTRO.br - Centro de Estudos e Pesquisas em Tecnologia de Redes e Operações. NTP.br - Network Time Protocol ou Protocolo de Tempo para Redes. CEWEB.br - Centro de Estudos sobre Tecnologias Web. CETIC.br - Centro Regional de Estudos para o Desenvolvimento da Sociedade da Informação.

\section{Quais características do protocolo HTTP descritas na RFC você já conhecia?}

O protocolo de transferência de hipertexto seguro (HTTPS).

\section{Qual o motivo de haver 2 chaves diferentes na figura do slide 32?}

As chaves são características do tipo de criptografia apresentado (End-to-end), sendo que a chave pública criptografa a mensagem enviada e a chave privada é usada para descriptografar a mesma. Somente com a chave privada é possível o acesso a mensagem enviada.

\end{document}
